\section{Podsumowanie}
Na podstawie uzyskanych wyników oceniamy, że próby przyspieszenia implementacji algorytmu przez zrównoleglenie się powiodły. Widać to wyraźnie na wykresach \ref{fig:chart_1}. i \ref{fig:chart_2}. Uzyskano dość dużą różnice czasu wykonywania programu przeznaczonego na \textbf{GPU} w stosunku do wersji \textbf{CPU}. Przyspieszenie było nawet 100-krotne  dla plików o rozmiarze 500 MB. Nie oznacza to jednak, że udało nam się osiągnąć pełnie możliwości przyspieszenia tego algorytmu. Wyniki analizy programu w wersji \textbf{GPU} przez narzędzia do optymalizacji algorytmu wykazują jego niską skuteczność w wykorzystaniu zasobów sprzętowych. 
Otrzymane wyniki pozwoliły nam lepiej zrozumieć mechanizmy działania, zalety i ograniczenia programowania na akceleratory graficzne. Szczególnie pouczająca okazała się analiza wyników narzędzia profilującego, gdyż do ich interpretacji wymagana jest głęboka znajomość budowy i usprawnień wykorzystywanych w architekturach \textbf{GPU}.